\documentclass[a4paper,10pt]{article}
\usepackage[croatian]{babel}

% tables
\usepackage{booktabs}
\usepackage{longtable}

% headers
\usepackage{fancyhdr}
\fancyhf{}
\usepackage[left=3cm,right=3cm,top=3cm,bottom=3cm]{geometry}

% toc font
\usepackage{tocloft}
%% naslov "sadrzaj"
\renewcommand{\cfttoctitlefont}{\Large\sc}
%% font stranica
\renewcommand{\cftsecpagefont}{\sc}
\renewcommand{\cftsubsecpagefont}{\sc}
%% imena poglavlja
\renewcommand{\cftsecfont}{\sc}
\renewcommand{\cftsubsecfont}{\sc}

%index
\usepackage{makeidx}
\makeindex

\lhead{\textsc{AIRbender VR}}
\rhead{\textsc{Lekcije za stvaranje igre}}
\cfoot{\thepage}

% side notes
\usepackage{todonotes}
\usepackage{xcolor}

\pagestyle{empty}
\linespread{1.3}

\usepackage{titlesec}
\titleformat*{\section}{\Large\sc}
\titleformat*{\subsection}{\large\sc}

\begin{document}

\begin{center}
\textsc{Sveučilište u Zagrebu}\\
\textsc{Fakultet organizacije i informatike}\\
\textsc{Analiza i razvoj programa}
\end{center}
\vspace*{7cm}
\begin{center}
	\textsc{\huge AIRbender VR}\\
	\textsc{\large Unreal Engine}\\Naputci za izradu\\
	\vspace*{1cm}
	Dorian Čajko \quad Ivan Huzjak \quad Denis Jocković \quad  Filip
	Novački \quad Luka Štefanić
\end{center}
\hspace*{1cm}

\begin{center}\end{center}
\vspace*{7cm}

\begin{center}
VARA\v{Z}DIN
\end{center}

\pagebreak

\pagestyle{fancy}
\tableofcontents

\pagebreak
\newgeometry{top=3cm, bottom=3cm, outer=6.5cm, inner=3cm, left=3cm,
marginparwidth=4.7cm, marginparsep=0.7cm}

\section*{Uvod}

Ovaj priručnik za programiranje projekta namijenjen je prvenstveno studentima
tehničkih područja kao nit vodilja kroz stvaranje jednog projekta. Igra koja se
stvara ovim putem temelji se na popularnoj crtanoj seriji Avatar, a ime, u
originalu \textit{Avatar, the Last Airbender}, vrlo se prikladno poklopio s
imenom kolegija, \textit{Analiza i razvoj programa}, iliti skraćeno
AIR.

\index{Uvod}
Podrazumijeva se da čitatelj ovog priručnika poznaje osnovne koncepte
programiranja te se oni ovdje neće u detalje objašnjavati. Naglasak će se
staviti na koncepte koji se upotrebljavaju u izradi igre, dakle oni vezani za
Unreal Engine, razvoj igara itd.

\marginpar{\color{blue}{Dodatna objašnjenja mogu se vidjeti izdvojena sa strane
kako bi se dodatno povezalo objašnjeno s drugim konceptima.}}
Glavni tekst sadržavat će objašnjenje postupaka koji se koriste u izradi igre.
Moguće je da se koraci malo promijene u određenim verzijama softvera koji se
koriste, no pretpostavlja se da će sve ostati slično jer se ne koriste jako
opskurni koncepti.

Na kraju priručnika nalaksi se kazalo pojmova kako bi se određeni pojam mogao
lakše pronaći ukoliko je samo spomenut negdje u tekstu, a nije iz naslova jasno
o kojem se pojmu točno radi.

\pagebreak


\section{Instalacija, pokretanje i snalaženje u okruženju }

\pagebreak
\section{Kreiranje i kretanje (teleportacija) glavnog lika}
\index{Vektor gledanja}
\index{Teleportacija}

\pagebreak
\section{Detekcija pokreta kontrolera}

\pagebreak
\section{Stvaranje meta}

\pagebreak
\section{Napadačke moći}

\pagebreak
\section{Stvaranje prijetnji za igrača i prikaz zdravlja (HP)}

\pagebreak
\section{Obrambeni mehanizmi za igrača}

\pagebreak
\section{Bodovanje i prikaz bodova}

\pagebreak
\section{Izrada krajolika nivoa}

\pagebreak
\section{Izbornici}

\pagebreak
\section{"Umjetna inteligencija" -- protivnik se kreće}

\pagebreak
\section{Vizualni efekti}

\pagebreak
\section{Zvuk}

\printindex
\end{document}
